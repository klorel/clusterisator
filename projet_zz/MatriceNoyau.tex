\documentclass[12pt]{article}
\usepackage{graphicx}
\usepackage[french]{babel}
\usepackage[T1]{fontenc}
\usepackage[applemac]{inputenc}
\usepackage{geometry}
\usepackage{amsmath}
\usepackage{amssymb}
\usepackage{tocloft}
\usepackage{wrapfig}
\usepackage{textcomp}
\usepackage{tikz}
\usepackage{caption}
\numberwithin{equation}{section}


\begin{document}

Soient $(V_1,...V_n) \in \mathbb R^q$ $n$ observations.
On note $K$ la matrice des produits scalaires entre ces observation, soit $K_{ij} = (V_i,V_j)$.

Notant $A$ la matrice dont les colonnes sont les vecteurs $V_i$, $1\leq i\leq n$,
soit $A = (V_1, ... V_n)$, on peut alors �crire :

\begin{center}

$A^TA = \left(  
\begin{matrix} 
      V_1^T \\
      . \\
      . \\
      . \\
      V_n^T           
   \end{matrix}
 \right)
 (V_1,...,V_n)
 =    \left(\begin{matrix} 
       (V_1,V_1) & ... & (V_1,V_n) \\
       . &  & . \\
       . &  & . \\
       . &  & . \\
       (V_n,V_1) & ... & (V_n,V_n)
    \end{matrix}
    \right)
    = K$
    
\end{center}

� partir de l�, notons $\lambda$ une valeur propre de $K$ et $X$ un vecteur propre associ�.
On a donc : 

\begin{center}

$KX = \lambda X$
\\[0.3em]
$X^TKX = \lambda X^TX$
\\[0.3em]
$X^TA^TAX = \lambda||X||^2$
\\[0.3em]
$(AX)^TAX = \lambda||X||^2$
\\[0.3em]
$||AX||^2 = \lambda||X||^2$

\end{center}

D'o� l'on d�duit : $\lambda \in \mathbb R^+$.
\\[2.5em]
De plus, si la matrice $K$ n'est pas inversible,
alors $ \exists a = (a_1,...a_k)$ tel que 
\begin{center}
$\forall j \displaystyle{\sum_{1\leq i\leq k}{a_iV_i\cdot V_j}} = 0$
\end{center}
Notant $w = \sum_{1\leq i\leq k}{a_iV_i}$, on trouve que $w$ est orthogonal � tous les $V_i$, donc � lui-m�me.

Donc $w$ est nul.

Donc les $V_i$ sont lin�airement d�pendants.
\\[1em]
R�ciproquement, si les $V_i$ sont lin�airement d�pendants, alors $K$ est clairement non-inversible.
\\[1em]
Donc $K$ est inversible si et seulement si les $V_i$ sont lin�airement ind�pendants.
\\[2.5em]
Donc $K$ est strictement d�finie positive si et seulement si les $V_i$ sont lin�airement ind�pendants.
Autrement, elle est seulement semi-d�finie positive.

\end{document}